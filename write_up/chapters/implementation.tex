%%%%%%%%%%%%%%%%%%%%%%%%%%%%%%%%%%%%%%%%%%%%%%%%%%%%%%%%%%%%%%%%%%%%%%%%%%%%%%%
% Chapter 4: Implementation
%%%%%%%%%%%%%%%%%%%%%%%%%%%%%%%%%%%%%%%%%%%%%%%%%%%%%%%%%%%%%%%%%%%%%%%%%%%%%%%

\section{Software Architecture}
\label{sec:impl_architecture}

\subsection{Overview}
\label{subsec:impl_overview}

The \texttt{QC-Amp} library is a Python package implementing the Chawdhry--Pellen algorithm for computing QCD colour factors. It is built on top of IBM's Qiskit framework~\cite{Qiskit2024} and provides a clean, well-documented API for constructing and evaluating colour factor circuits.

The package follows a modular design with clear separation of concerns:

\begin{itemize}
    \item \texttt{su3.py}: SU(3) group theory utilities (Gell-Mann matrices, structure constants)
    \item \texttt{gates.py}: Quantum gate definitions ($A$, $B$, $\Lambda$, $M$, $Q$, $G$)
    \item \texttt{circuits.py}: High-level circuit builders for specific diagrams
    \item \texttt{colour\_factors.py}: Colour factor computation and verification
\end{itemize}

\subsection{Directory Structure}
\label{subsec:directory_structure}

\begin{lstlisting}[language=bash,caption={Package directory structure},label={lst:directory}]
QC-Amp/
+-- src/
|   +-- qc_amp/
|       +-- __init__.py
|       +-- su3.py           # SU(3) utilities
|       +-- gates.py         # Gate definitions
|       +-- circuits.py      # Circuit builders
|       +-- colour_factors.py # Colour factor computation
+-- tests/
|   +-- test_su3.py
|   +-- test_gates.py
|   +-- test_circuits.py
|   +-- test_colour_factors.py
+-- examples/
|   +-- colour_factors_demo.ipynb
+-- pyproject.toml
+-- README.md
\end{lstlisting}

\subsection{Dependencies}
\label{subsec:dependencies}

The package requires the following dependencies:
\begin{itemize}
    \item Python $\geq 3.9$
    \item NumPy $\geq 1.21$
    \item Qiskit $\geq 1.0$
\end{itemize}

Installation is performed via pip:
\begin{lstlisting}[language=bash]
pip install -e .
\end{lstlisting}


\section{Module: \texttt{su3.py}}
\label{sec:impl_su3}

\subsection{Purpose}
\label{subsec:su3_purpose}

The \texttt{su3.py} module provides the mathematical foundations for SU(3) colour calculations. It defines the Gell-Mann matrices, unitary-adjusted matrices, and structure constants.

\subsection{Gell-Mann Matrices}
\label{subsec:impl_gellmann}

The eight Gell-Mann matrices are defined as NumPy arrays:

\begin{lstlisting}[language=Python,caption={Gell-Mann matrix definitions (excerpt)},label={lst:gellmann}]
import numpy as np

L1 = np.array([
    [0, 1, 0],
    [1, 0, 0],
    [0, 0, 0]
], dtype=complex)

# ... (L2 through L7 similarly)

L8 = (1 / np.sqrt(3)) * np.array([
    [1, 0, 0],
    [0, 1, 0],
    [0, 0, -2]
], dtype=complex)

GELL_MANN_MATRICES = [L1, L2, L3, L4, L5, L6, L7, L8]
\end{lstlisting}

\subsection{Unitary-Adjusted Matrices}
\label{subsec:impl_adjusted}

The unitary-adjusted matrices $\lhat{a}$ are constructed following the procedure described in Section~\ref{sec:algo_unitary_adjusted}:

\begin{lstlisting}[language=Python,caption={Unitary-adjusted matrix definitions (excerpt)},label={lst:adjusted}]
# Example: l1 is lambda_1 with (3,3) element set to 1
l1 = np.array([
    [0, 1, 0],
    [1, 0, 0],
    [0, 0, 1]  # Changed from 0 to 1 for unitarity
], dtype=complex)

# Special case: l8 is the identity
l8 = np.array([
    [1, 0, 0],
    [0, 1, 0],
    [0, 0, 1]
], dtype=complex)

UNITARY_ADJUSTED_MATRICES = [l1, l2, l3, l4, l5, l6, l7, l8]
\end{lstlisting}

\subsection{Matrix Expansion}
\label{subsec:impl_expand}

Since quark colours are encoded in 2 qubits (4-dimensional Hilbert space), the $3 \times 3$ matrices must be embedded into $4 \times 4$ matrices:

\begin{lstlisting}[language=Python,caption={Matrix expansion function},label={lst:expand}]
def expand_matrix(mat: np.ndarray) -> np.ndarray:
    """
    Embed a 3x3 matrix into 4x4 by adding an unused |11> level.
    
    The |11> state is left invariant (acts as identity on this level).
    """
    N = mat.shape[0]  # should be 3
    expanded = np.zeros((N + 1, N + 1), dtype=complex)
    expanded[:N, :N] = mat
    expanded[N, N] = 1  # Identity on |11>
    return expanded
\end{lstlisting}

\subsection{Structure Constants}
\label{subsec:impl_structure}

The SU(3) structure constants $\structconst{a}{b}{c}$ are computed from the commutation relations:

\begin{lstlisting}[language=Python,caption={Structure constant computation},label={lst:structure}]
def su3_structure_constants():
    """Compute the SU(3) structure constants f_abc."""
    Ts = [0.5 * L for L in GELL_MANN_MATRICES]  # T_a = lambda_a / 2
    
    f = {}
    for a in range(8):
        for b in range(8):
            comm = Ts[a] @ Ts[b] - Ts[b] @ Ts[a]  # [T_a, T_b]
            for c in range(8):
                # f_abc = -2i * Tr([T_a, T_b] * T_c)
                val = (-2j) * np.trace(comm @ Ts[c])
                if abs(val) > 1e-10:
                    f[(a+1, b+1, c+1)] = val
    return f
\end{lstlisting}


\section{Module: \texttt{gates.py}}
\label{sec:impl_gates}

\subsection{The R Gate}
\label{subsec:impl_r_gate}

The $R$ gate prepares the quark colour superposition state:

\begin{lstlisting}[language=Python,caption={R gate definition},label={lst:r_gate}]
R_MATRIX = np.array([
    [np.sqrt(1/3),  np.sqrt(1/2), -np.sqrt(1/6), 0],
    [np.sqrt(1/3), -np.sqrt(1/2), -np.sqrt(1/6), 0],
    [np.sqrt(1/3),  0,             np.sqrt(2/3), 0],
    [0,             0,             0,            1],
], dtype=complex)

R_GATE = UnitaryGate(R_MATRIX, label="R")
\end{lstlisting}

\subsection{The A Gate}
\label{subsec:impl_a_gate}

The increment operator is implemented using a cascade of multi-controlled X gates:

\begin{lstlisting}[language=Python,caption={A gate implementation},label={lst:a_gate}]
def A_gate(register: QuantumRegister) -> QuantumCircuit:
    """Increment operator A on the unitarisation register."""
    n = len(register)
    qc = QuantumCircuit(register, name="A")
    
    # Multi-controlled X gates for ripple-carry increment
    for i in range(n - 1):
        controls = [register[j] for j in range(n - i - 1)]
        target = register[n - i - 1]
        qc.mcx(controls, target)
    
    qc.x(register[0])  # LSB always flips
    return qc
\end{lstlisting}

\subsection{The B Gate}
\label{subsec:impl_b_gate}

The amplitude transfer gate consists of the single-qubit $B_1$ rotation and its controlled version:

\begin{lstlisting}[language=Python,caption={B gate implementation},label={lst:b_gate}]
def B1_gate(alpha: complex) -> UnitaryGate:
    """Single-qubit B1(alpha) rotation gate."""
    a = complex(alpha)
    if abs(a)**2 > 1 + 1e-12:
        raise ValueError(f"|alpha|^2 must be <= 1")
    
    s = np.sqrt(1 - abs(a)**2)
    mat = np.array([
        [s, a],
        [-np.conj(a), s]
    ], dtype=complex)
    
    return UnitaryGate(mat, label=f"B1({alpha:.3f})")


def B_gate(alpha: complex, U: QuantumRegister) -> QuantumCircuit:
    """B(alpha) gate controlled on U[1:] being |0...0>."""
    NU = len(U)
    qc = QuantumCircuit(U, name=f"B({alpha:.3f})")
    
    if NU == 1:
        qc.append(B1_gate(alpha), [U[0]])
        return qc
    
    base = B1_gate(alpha)
    ctrl_state = "0" * (NU - 1)
    controlled = base.control(num_ctrl_qubits=NU-1, ctrl_state=ctrl_state)
    qc.append(controlled, list(U[1:]) + [U[0]])
    
    return qc
\end{lstlisting}

\subsection{The \texorpdfstring{$\mu$}{μ} Coefficient}
\label{subsec:impl_mu}

The correction coefficient function encodes the logic described in Section~\ref{sec:algo_mu}:

\begin{lstlisting}[language=Python,caption={$\mu$ coefficient computation},label={lst:mu}]
def mu_coefficient(a: int, i: int) -> complex:
    """Compute mu(a,i) coefficient for the M operator."""
    l_unit = UNITARY_ADJUSTED_MATRICES[a - 1]
    l_orig = GELL_MANN_MATRICES[a - 1]
    
    # Check if the i-th row matches
    row_equal = np.allclose(l_unit[i-1, :], l_orig[i-1, :], atol=1e-10)
    
    if row_equal:
        return 0.5  # Standard T = lambda/2 normalization
    
    # For differing rows, return half the diagonal element
    orig_diag = l_orig[i - 1, i - 1]
    return complex(orig_diag) / 2
\end{lstlisting}

\subsection{The \texorpdfstring{$\Lambda$}{Lambda} Gate}
\label{subsec:impl_lambda_gate}

The gluon-controlled colour rotation:

\begin{lstlisting}[language=Python,caption={$\Lambda$ gate implementation},label={lst:lambda_gate}]
def Lambda_gate(quark: QuantumRegister, gluon: QuantumRegister):
    """Lambda gate: gluon-controlled colour rotation on quark."""
    qc = QuantumCircuit(quark, gluon, name="Lambda")
    n_ctrl = len(gluon)
    
    for a, L in enumerate(UNITARY_ADJUSTED_MATRICES, start=1):
        expanded = expand_matrix(L)
        base_gate = UnitaryGate(expanded, label=f"lhat{a}")
        
        # Control on gluon state |a-1>
        ctrl_state = format(a - 1, f"0{n_ctrl}b")
        CU = base_gate.control(num_ctrl_qubits=n_ctrl, 
                               ctrl_state=ctrl_state)
        
        qc.append(CU, gluon[:] + quark[:])
    
    return qc
\end{lstlisting}

\subsection{The M Gate}
\label{subsec:impl_m_gate}

The unitarisation correction operator:

\begin{lstlisting}[language=Python,caption={M gate implementation},label={lst:m_gate}]
def M_gate(gluon, quark, U):
    """M operator for unitarisation correction."""
    qc = QuantumCircuit(gluon, quark, U, name="M")
    n_g, n_q = len(gluon), len(quark)
    
    for a in range(1, 9):      # gluon colours 1-8
        for i in range(1, 4):  # quark colours 1-3
            alpha = mu_coefficient(a, i)
            
            if abs(alpha) < 1e-12:
                continue
            
            B_circ = B_gate(alpha, U)
            B_gate_obj = B_circ.to_gate()
            
            # Compute control state: (a-1) + 8*(i-1)
            gluon_state = a - 1
            quark_state = i - 1
            ctrl_state_int = gluon_state + (2**n_g) * quark_state
            ctrl_state = format(ctrl_state_int, f"0{n_g + n_q}b")
            
            controlled = B_gate_obj.control(
                num_ctrl_qubits=n_g + n_q,
                ctrl_state=ctrl_state
            )
            qc.append(controlled, list(gluon) + list(quark) + list(U))
    
    return qc
\end{lstlisting}

\subsection{The Q Gate}
\label{subsec:impl_q_gate}

The complete quark-gluon vertex:

\begin{lstlisting}[language=Python,caption={Q gate implementation},label={lst:q_gate}]
def Q_gate(gluon, quark, U):
    """Q gate: complete quark-gluon interaction vertex."""
    qc = QuantumCircuit(gluon, quark, U, name="Q")
    
    # 1. Increment U with A
    qc.compose(A_gate(U), qubits=U, inplace=True)
    
    # 2. Conditional correction via M
    qc.compose(
        M_gate(gluon, quark, U),
        qubits=list(gluon) + list(quark) + list(U),
        inplace=True
    )
    
    # 3. Apply Lambda on (quark, gluon)
    qc.compose(
        Lambda_gate(quark, gluon),
        qubits=list(quark) + list(gluon),
        inplace=True
    )
    
    return qc
\end{lstlisting}


\section{Module: \texttt{circuits.py}}
\label{sec:impl_circuits}

\subsection{Quark-Antiquark Preparation}
\label{subsec:impl_quark_prep}

\begin{lstlisting}[language=Python,caption={Quark singlet preparation},label={lst:quark_prep}]
def R_quark_prep(quark, anti_quark):
    """Prepare quark-antiquark pair in colour singlet-like state."""
    qc = QuantumCircuit(quark, anti_quark, name="R_q")
    
    # Apply R gate to quark register
    qc.append(R_GATE, quark)
    
    # Entangle antiquark with quark via CNOTs
    qc.cx(quark[0], anti_quark[0])
    qc.cx(quark[1], anti_quark[1])
    
    return qc
\end{lstlisting}

\subsection{Gluon Preparation}
\label{subsec:impl_gluon_prep}

\begin{lstlisting}[language=Python,caption={Gluon superposition preparation},label={lst:gluon_prep}]
def R_gluon_prep(gluon):
    """Prepare gluon in equal superposition over 8 colours."""
    qc = QuantumCircuit(gluon, name="R_g")
    qc.h(gluon)  # H^3 creates equal superposition
    return qc
\end{lstlisting}

\subsection{Quark Emission-Absorption Circuit}
\label{subsec:impl_emission_absorption}

\begin{lstlisting}[language=Python,caption={Quark self-energy circuit},label={lst:self_energy_circuit}]
def quark_emission_absorption(n_vertices: int = 2):
    """Build circuit for quark emitting and absorbing a gluon."""
    NU = int(np.ceil(np.log2(n_vertices) + 1))
    
    # Create registers
    g = QuantumRegister(3, 'g')        # gluon
    U = QuantumRegister(NU, 'U')       # unitarisation
    q = QuantumRegister(2, 'q')        # quark
    qtil = QuantumRegister(2, 'qbar')  # anti-quark
    
    qc = QuantumCircuit(g, U, q, qtil, name="QuarkGluonDiagram")
    
    # Build gates
    Rg_gate = R_gluon_prep(g).to_gate(label="R_g")
    Rq_gate = R_quark_prep(q, qtil).to_gate(label="R_q")
    Q_gate_obj = Q_gate(g, q, U).to_gate(label="Q")
    
    # Circuit structure: R_g -> R_q -> Q^n -> R_g^dag -> R_q^dag
    qc.append(Rg_gate, g)
    qc.append(Rq_gate, list(q) + list(qtil))
    
    for _ in range(n_vertices):
        qc.append(Q_gate_obj, list(g) + list(q) + list(U))
    
    qc.append(Rg_gate.inverse(), g)
    qc.append(Rq_gate.inverse(), list(q) + list(qtil))
    
    return qc
\end{lstlisting}


\section{Module: \texttt{colour\_factors.py}}
\label{sec:impl_colour_factors}

\subsection{Colour Factor Computation}
\label{subsec:impl_compute}

\begin{lstlisting}[language=Python,caption={Colour factor computation},label={lst:compute_colour_factor}]
def compute_colour_factor_detailed(circuit, n_quarks=1, n_gluons=1, N_c=3):
    """
    Compute the colour factor with detailed intermediate results.
    
    Returns:
        Tuple of (C, amplitude, N) where:
            - C: The colour factor
            - amplitude: The raw <0|psi_final> amplitude
            - N: The normalization factor
    """
    # 1. Prepare initial state |0...0>
    psi0 = Statevector.from_label("0" * circuit.num_qubits)
    
    # 2. Evolve through the circuit
    psi_final = psi0.evolve(circuit)
    
    # 3. Extract amplitude of |0...0> (first component)
    amp_omega = psi_final.data[0]
    
    # 4. Compute normalization factor
    N = (N_c ** n_quarks) * ((N_c**2 - 1) ** n_gluons)
    
    # 5. Colour factor
    C = N * amp_omega
    
    return C, amp_omega, N
\end{lstlisting}


\section{Testing}
\label{sec:impl_testing}

\subsection{Test Suite Overview}
\label{subsec:test_overview}

The \texttt{QC-Amp} library includes a comprehensive test suite using the pytest framework. Tests are organised into modules corresponding to the source modules:

\begin{itemize}
    \item \texttt{test\_su3.py}: Tests for Gell-Mann matrices and structure constants
    \item \texttt{test\_gates.py}: Tests for individual gate unitarity and correctness
    \item \texttt{test\_circuits.py}: Tests for circuit construction
    \item \texttt{test\_colour\_factors.py}: End-to-end tests for colour factor computation
\end{itemize}

\subsection{Key Test Cases}
\label{subsec:test_cases}

\begin{lstlisting}[language=Python,caption={Example test cases},label={lst:tests}]
def test_gellmann_hermitian():
    """All Gell-Mann matrices should be Hermitian."""
    for i, L in enumerate(GELL_MANN_MATRICES):
        assert np.allclose(L, L.conj().T), f"L{i+1} not Hermitian"


def test_adjusted_unitary():
    """All adjusted matrices should be unitary."""
    for i, L in enumerate(UNITARY_ADJUSTED_MATRICES):
        product = L.conj().T @ L
        assert np.allclose(product, np.eye(3)), f"l{i+1} not unitary"


def test_colour_factor_self_energy():
    """Quark self-energy colour factor should be 4."""
    circuit = quark_emission_absorption(n_vertices=2)
    C, amp, N = compute_colour_factor_detailed(circuit, n_quarks=1, n_gluons=1)
    assert np.isclose(C.real, 4.0, atol=1e-6), f"Expected C=4, got {C}"
\end{lstlisting}

\subsection{Running Tests}
\label{subsec:running_tests}

Tests are run using pytest:
\begin{lstlisting}[language=bash]
cd QC-Amp
pytest tests/ -v
\end{lstlisting}

All tests pass successfully, confirming the correctness of the implementation.
